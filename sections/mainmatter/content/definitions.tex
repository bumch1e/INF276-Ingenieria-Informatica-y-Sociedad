\section*{Definiciones}

Entrando en materia, es bueno conocer ciertos conceptos de
los que se estará hablando con regularidad.

\subsection*{Persona}
Individuo de la especie humana. Sujeto de derecho. Supuesto
inteligente.

\subsection*{Sociedad}
Conjunto de personas, pueblos o naciones que conviven bajo
normas comunes. Agrupación natural o pactada de personas,
organizada para cooperar en la consecución de determinados
fines.

\subsection*{Tecnología}
Conjunto de teorías y de técnicas que permiten el
aprovechamiento práctico del conocimiento científico.

\subsection*{Ciencia}
Conjunto de conocimientos obtenidos mediante la observación
y el razonamiento, sistemáticamente estructurados y de los
que se deducen principios y leyes generales con capacidad
predictiva y comprobables experimentalmente. Habilidad,
maestría, conjunto de conocimientos en cualquier cosa.\\

De ella, se deduce: \\

\textbf{Teoría:} Explicación de un aspecto del mundo natural
que puede ser o ha sido comprobada repetidas veces y es
corroborada por evidencia. \\

\textbf{Hipótesis:} Suposición o explicación propuesta hecha
en base a evidencia limitada como punto de partida para una
investigación futura.

\subsection*{Ingeniería}
Conjunto de conocimientos orientados a la invención y
utilización de técnicas para el aprovechamiento de los
recursos naturales o para la actividad industrial. \\

La ingeniería, de forma más general, es el diseño, prueba y
construcción de máquinas, estructuras y procesos usando
matemáticas y ciencias buscando soluciones eficaces a
problemas diversos.

\subsection*{Ingeniero}
Persona con titulación universitaria que la capacita para
ejercer la ingeniería.

\subsection*{Informática}
Conjunto de conocimientos científicos y técnicas que hacen
posible el tratamiento automático de la información por
medio de computadoras.\\

Es una discplina \textit{sui generis} y no tiene soporte
directo a diferencia de otras ciencias.\\

El ramo, en síntesis, busca hablar sobre cómo la ingeniería
informática incide en la sociedad y las responsabilidades
del ingeniero en ella.
