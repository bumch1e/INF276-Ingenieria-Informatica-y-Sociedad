\section*{Sesgos cognitivos}

Un \textbf{sesgo cognitivo} es un patrón sistemático de
desviarse de la norma o la racionalidad al momento de
juzgar. Se refiere a la distorsción de la percepción y
realidad de un individuo en particular. \\

Los sesgos cognitivos se pueden notar en distintas
dimensiones:

\begin{itemize}
\item{\textbf{Específicos a grupos:} Polarización frente
        a otras posiciones individuales.}
  \item{\textbf{Toma de decisiones:} El deseo del resultado
        es tentador.}
  \item{\textbf{Correlación ilusoria:} Relación errónea de
        variables independientes.}
  \item{\textbf{Memoria:} Recordatorio constante del pasado.}
  \item{\textbf{Motivación: }Deseo de autoimagen positiva.}
\end{itemize}

En base a estos criterios, se puede definir una lista parcial
de sesgos cognitivos.

\begin{itemize}
\item{\textbf{Recordar preferentemente lo positivo.}}
\item{\textbf{Preferir los resultados positivos.}}
\item{\textbf{Generalizar a grupos}}
\item{\textbf{Representatividad (de ideales).}}
\item{\textbf{Error fundamental de atribución.} Predisposición.}
\item{\textbf{Sesgo de imprimación.} Primera impresión.}
\item{\textbf{Sesgo de confirmación.} Interpretar lo que se quiere interpretar.}
  \item{\textbf{Sesgo de afinidad.} Favorecer a los que se
        parecen a nosotros.}
  \item{\textbf{Sesgo de interés propio.} Sí a los éxitos, no
        a los fracasos.}
  \item{\textbf{Sesgo de creencia.} Evaluar el razonamiento
        en base a creencias.}
\item{\textbf{Status quo:} Dejar las cosas como están.}
\item{\textbf{Confianza excesiva:} Dunning-Kruger.}
  \item{\textbf{Estereotipo de atractivo físico:} La belleza
        no está en el interior?\ldots}
  \item{\textbf{Efecto halo:} Impresiones que contaminan
        otras evaluaciones.}
  \item{\textbf{Disponibilidad.} Evaluar información en base
        a lo reciente o más fácil de recordar.}
\item{\textbf{Tribalismo.} Alienarse en tribus (urbanas).}
\item{\textbf{Causa popular.} Tendencia a lo popular.}
  \item{\textbf{Costo hundido.} Que valga la pena lo
        invertido.}
\end{itemize}

Bajo esta ideas, surge también la \textbf{disonancia
  cognitiva}. La disonancia cognitiva es un fenómeno
psicológico que se presenta cuando una persona mantiene
dos creencias contradictorias a la vez. \\

Una persona con disonancia cognitiva tiende a defenderse
de lo que causa incomodidad \textbf{evitando, deslegitimando
  y limitando su impacto}. \\

La única forma de superar los sesgos es mantener buenos
hábitos. Hay que ser \textbf{objetivos, humildes y críticos
de nosotros mismos}.
