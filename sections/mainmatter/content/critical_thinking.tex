\section*{Pensamiento crítico}

El pensamiento crítico consiste en cuestionar, analizar,
interpretar, evaluar y juzgar de forma confiable basándose
en información fidedigna. \\

Las habilidades que requiere el pensamiento crítico son:

\begin{itemize}
  \item{Clarificar el propósito y contexto.}
  \item{Cuestionar las fuentes de información.}
  \item{Identificar argumentos.}
  \item{Analizar fuentes y argumentos.}
  \item{Evaluar argumetnos ajenos.}
  \item{Crear o sintetizar argumentos propios.}
\end{itemize}

Dentro del espectro, tambień hay otros componentes que
rodean al concepto de pensamiento crítico. Uno de estos
es el \textbf{pensamiento analítico}, que consiste en
evaluar datos de múltiples fuentes y contrastar información
para obtenr las mejores conclusiones. \\

Es importante \textbf{mantener la mente abierta} con toda
la información nueva y considerar puntos de vista alternos
al propio.

Es importante también cultivar la \textbf{resolución de
  problemas}, pues permite reconocer problemas de manera
clara y darles una solución adecuado dentro del contexto
donde surjen en base a un  \textbf{juicio razonado}.
