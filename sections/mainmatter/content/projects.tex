\section*{Proyectos}

Para empezar, un lineamiento ``general'' de etapas para
cualquier proyecto.

\begin{enumerate}
  \item{Optimismo general.}
  \item{Fase de desorientación.}
  \item{Inicio a trancazos de las obras.}
  \item{Período de despelote.}
  \item{Búsqueda implacable de culpables.}
  \item{Sálvese quien pueda.}
  \item{Castigo ejemplar a los inocentes.}
  \item{Recuperación del optimismo.}
  \item{Terminación inexplicable del proyecto.}
  \item{Condecoración y felicitaciones a los no
  participantes.}
\end{enumerate}

En la realidad, es una posibilidad no despreciable que un
proyecto falle en su quehacer. Un proyecto se considera
fallido si:

\begin{itemize}
\item{No cumplió los objetivos.}
\item{No se obtuvo el producto buscado.}
\item{El trabajo no se completó a tiempo.}
\end{itemize}

Tmabién existen problemas a los que que todo proyecto puede
enfrentarse si no se tiene el cuidado suficiente:

\begin{itemize}
\item \textbf{Objetivos poco claros.} Los objetivos son el
  fin último de cada proyecto. Deben ser medibles, con metas
  de tiempo y específicos.
\item \textbf{Scope Creep.} La ampliación del alcance (o la
  escalabilidad) es lenta y propensa a errores. Se debe
  trabajar con un ambiente de trabajo claro y documentado
  para todas las partes involucradas.
\item \textbf{Expectativas poco realistas.} Metas demasiado
  ambiciosas pueden recaer en problemas de estrés y plazos
  sin cumplir.
\item \textbf{Recursos limitados.} Es vital que un proyecto
  siempre cuente con los recursos mínimos para continuar
  su desarrollo.
\item \textbf{Comunicaciones pobres.} Es primordial que
  exista comunicación entre los implicados. Coincidir en
  un canal de comunicación claro y común es la clave
  para no perder el hilo del desarrollo.
\item \textbf{Retardos de agenda.} Una programación
  detallada es la clave para evitar el no cumplimiento
  de hitos o reuniones.
\item \textbf{Falta de transparencia.} La documentación es
  importante en cualquier tipo de proyecto. Si no existe,
  la comunicación de actualizaciones o el valor de la propia
  información se pierde.
\end{itemize}

Teniendo en consideración todas las problemáticas que pueden
aquejar a un proyecto, es bueno considerar aquello que nos
puede ayudar a que el proyecto sobreviva y se desarrolle
con creces. Algunas directrices son:

\begin{itemize}
\item{Tener un trabajo en grupo bien consolidado.}
\item{Establecer expectivas realistas (Definición de
objetivos, fechas límite, roles y responsabilidades).}
\item{Reuniones exitosas (Límites claros, uso de agenda,
  planteamiento de resultados, minutas escritas, ítem de
  acción claros.)}
\item{Presentaciones decentes (preparación, uso apropiado,
audiencia, consultas, no confiar en la tecnología.}
\end{itemize}
