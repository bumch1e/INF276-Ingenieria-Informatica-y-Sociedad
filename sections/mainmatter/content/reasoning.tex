\section*{Razonamiento}

Cuando se habla de razonamiento, se tiene en cuenta a la
lógica de manera instintiva. \\

Por ejemplo, se dice que un argumento es válido si concluye
en base a premisas ad hoc al tema que se está discutiendo.
Un argumento de este tipo se conoce como
\textit{deductivamente válido}. \\

Por otro lado, un argumento se dice sólido si es válido y
sus premisas son verdaderas. \\

Siguiendo con el tema, hay veces en las que un argumento
se cae por diversas razones y no es posible considerarlo
como válido y menos sólido. Estas caídas se deben al uso
de las llamadas \textbf{falacias}. Una falacia es un paso
ilógico en la argumentación. \textbf{OJO.} Que un argumento
recurra en falacias no implica que su conclusión sea falsa
per se. \\

Las falacias están insmicuidas en muchas partes de la vida
cotidiana. Algunos ejemplos reconocidos son:

\begin{itemize}
  \item{\textbf{Ad hominem:} Desacreditación hacia el
        contrincante.}
  \item{\textbf{Ad verecundiam:} Apelar a la autoridad.}
  \item{\textbf{Ad baculum:} Convencer por la fuerza.}
  \item{\textbf{Ad consequentiam: } Ceder por creencias.}
  \item{\textbf{Mundo justo:} Pasa porque el mundo es justo.}
  \item{\textbf{Apelar al miedo:} El nombre lo dice.}
  \item{\textbf{Apelar a la simpatía:} El nombre lo dice.}
  \item{\textbf{Teleológica:} Todo es un propósito para un
        fin.}
  \item{\textbf{Ad populum:} Apelar a lo popular.}
  \item{\textbf{Petitio principii:} Repetir una premisa.}
  \item{\textbf{Falso dilema:} Asumir solo dos opciones.}
  \item{\textbf{Falsa analogía:} Nada que ver.}
  \item{\textbf{Afirmar al consecuente: } Si P implica Q,
        entonces Q implica P.}
  \item{\textbf{Suprimir evidencia:} Cherry picking.}
  \item{\textbf{Estadísticas defectuosas:} El nombre lo
        dice.}
  \item{\textbf{Generalización apresurada: }El nombre lo
        dice.}
  \item{\textbf{Ignorar evidencia: }El nombre lo dice.}
  \item{\textbf{Rótulos cargados: }Apelar a lo emocional.}
  \item{\textbf{Non sequitur: }Nada que ver la conclusión.}
  \item{\textbf{Post hoc, ergo propter hoc: }Relación
        temporal no es una causal.}
  \item{\textbf{Pista falsa: }Red herring.}
  \item{\textbf{Mover la carga de la prueba: }Que argumente
        el otro.}
  \item{\textbf{Pendiente resbaladiza: }Todo se relaciona
        hacia el inminente desastre.}
  \item{\textbf{Hombre de paja: }Ridiculizar la posición
        contraria.}
  \item{\textbf{Olvidar lo central: }El nombre lo dice.}
  \item{\textbf{Apelar a la ignorancia: }El nombre lo dice.}
  \item{\textbf{Equívoco:} Manipular términos confusos.}
  \item{\textbf{Falso escocés: }Desestimar ejemplos.}
  \item{\textbf{Del jugador: }Ahora sí pasa.}
  \item{\textbf{Del alegato especial: }La excepción es válida.}
  \item{\textbf{Genético: }El origen es irrelevante.}
\end{itemize}
